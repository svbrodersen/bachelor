
\begin{frame}[hoved]
  \frametitle{Background}
  \begin{minipage}{0.45\textwidth}
  {\large Accelerator-based Computer Architecture}
  \begin{itemize}
    \item Off-loading the CPU.
    \item Optimized for distinct objectives, instead of General Purpose.
    \item Prominent example is the Graphical Processing Unit(GPU).
  \end{itemize}
  \vspace{0.5em}
  {\large RISC-V}
  \begin{itemize}
    \item Reduced Instruction set Computing(RISC), version 5 (V).
    \item Open source, minimize intellectual property, reduce barrier of entry.
    \item Provides RV32I, RV64I and RV128I.
  \end{itemize}
\end{minipage}
\hfill
\begin{minipage}{0.45\textwidth}
  \begin{figure}
    \begin{center}
      \includegraphics[width=0.95\textwidth]{figures/RISC-V-logo.png}
    \end{center}
    \caption{RISC-V logo \\ \tiny By ™/®RISC-V Foundation - Vectorised by
    Vulphere from
  https://riscv.org/wp-content/uploads/2017/05/Tue1100-RISC-V-Foundation-Update.pdf,
Public Domain,
https://commons.wikimedia.org/w/index.php?curid=69489388}\label{fig:riscv}
  \end{figure}
\end{minipage}
\end{frame}

\begin{frame}[hoved]
  \frametitle{Background}
  \begin{minipage}{0.45\textwidth}
  {\large QEMU}
  \begin{itemize}
    \item Able to emulate both a 32-bit and 64-bit RISC-V processor.
    \item Made it possible to develop a binary file targeting RISC-V working on
      another ISA.
  \end{itemize}
  \vspace{0.5em}
  {\large LLVM}
  \begin{itemize}
    \item A collection of reusable compiler and toolchain technologies.
    \item LLVM allows for the use of an LSP.
    \item Includes a cross-compiler capable of targeting RISC-V, but without
      newlib.
  \end{itemize}
\end{minipage}
\hfill
\begin{minipage}{0.45\textwidth}
  \begin{figure}
    \begin{center}
      \includegraphics[width=0.75\textwidth]{figures/LLVM-logo.png}
    \end{center}
    \caption{LLVM-logo \\
    \tiny https://www.llvm.org/Logo.html}\label{fig:llvm-logo}
  \end{figure}
\end{minipage}
\end{frame}

