
\begin{frame}[hoved]
	\frametitle{Background}
	\begin{minipage}[t]{0.45\textwidth}
		{\large Accelerator-based Computer Architecture}
		\begin{itemize}
			\item Off-loading the CPU.
			\item Optimized for distinct objectives, instead of General Purpose.
			\item Prominent example is the Graphical Processing Unit(GPU).
		\end{itemize}
		{\large RISC-V}
		\begin{itemize}
			\item Reduced Instruction set Computing(RISC), version 5 (V).
			\item Open source, minimize intellectual property, reduce barrier of entry.
			\item Provides RV32I, RV64I and RV128I.
		\end{itemize}
	\end{minipage}
	\hfill
	\begin{minipage}[t]{0.45\textwidth}
		\begin{figure}
			\begin{center}
				\includegraphics[width=0.95\textwidth]{figures/RISC-V-logo.png}
			\end{center}
			\caption{RISC-V logo. \tiny By RISC-V Foundation - Vectorised by
				Vulphere from
				https://riscv.org/wp-content/uploads/2017/05/Tue1100-RISC-V-Foundation-Update.pdf,
				Public Domain,
				https://commons.wikimedia.org/w/index.php?curid=69489388}\label{fig:riscv}
		\end{figure}
	\end{minipage}
\end{frame}

\begin{frame}[hoved]
	\frametitle{Background}
	\begin{minipage}[t]{0.45\textwidth}
		{\large QEMU}
		\begin{itemize}
			\item Able to emulate both a 32-bit and 64-bit RISC-V processor.
			\item Made it possible to develop a binary file targeting RISC-V working on
			      another ISA.
		\end{itemize}
		\vspace{0.5em}
	\end{minipage}
	\hfill
	\begin{minipage}[t]{0.45\textwidth}
		\begin{figure}
			\begin{center}
				\includegraphics[width=0.75\textwidth]{figures/QEMU-logo.png}
			\end{center}
			\caption{QEMU-logo.
				\tiny https://www.logo.wine/logo/QEMU}\label{fig:qemu-logo}
		\end{figure}
	\end{minipage}
\end{frame}

