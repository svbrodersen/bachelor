\section{Implementation}
The implementation aims to carry out the design presented in
Section~\ref{sec:Design}. First I will go through the Dependencies needed for
development. Next, I highlight a few components of the implementations, such
that the reader can get a better understanding of the implementation. Note, the
elements presented within this thesis is just a part of the implementation and
more parts make up the entire program.

\subsection{Dependencies}
\subsubsection*{QEMU}
\begin{lstlisting}[caption=Installing QEMU, label=lst:qemu_install, float=h]
git clone git clone https://github.com/qemu/qemu  # Clone the QEMU repo
cd qemu
./configure --target-list=riscv32-softmmu  # Configure the 32-bit RISC-V target
make -j $(nproc)  # build the project with all num cores jobs
sudo make install
\end{lstlisting}
Following the instructions from RISC-V's getting started guide, I can build the
QEMU RISC-V system emulators by running the code provided in
Listing~\ref{lst:qemu_install}\cite{RISC-V_GS}.

\subsubsection*{LLVM and RISC-V-gnu-toolchain}
\begin{lstlisting}[caption=Installing LLVM compiler infastructure with RISC-V
32-bit as native target., label=lst:llvm_install, float=h]
# Dependencies
sudo apt-get -y install \
  binutils build-essential libtool texinfo \
  gzip zip unzip patchutils curl git \
  make cmake ninja-build automake bison flex gperf \
  grep sed gawk python bc \
  zlib1g-dev libexpat1-dev libmpc-dev \
  libglib2.0-dev libfdt-dev libpixman-1-dev

git clone https://github.com/riscv-collab/riscv-gnu-toolchain
cd riscv-gnu-toolchain  # change directory
./configure --prefix=/opt/riscv --with-arch=rv32gc -disable-linux --enable-llvm
sudo make -j$(nproc)
cd ..
popd
\end{lstlisting}
Although the LLVM Clang compiler comes with an available cross-compiler, I found
that it often caused issues with missing header files compatible with my
implementation. Furthermore, the LLDB debugger was unable to provide a working
debugger for multicore remote debugging on QEMU. These are issues which might
only be affecting me, as information revolving around the issues was scarce. As
such, the following steps of building LLVM and the RISC-V 32-bit GDB may be
obsolete but are left here as a known working toolchain. Running the code in
Listing~\ref{lst:llvm_install} installs a RISC-V compatible Clang compiler and
GDB debugger in the /opt/riscv/ directory. For use outside this folder, make
sure to add it to PATH.

\subsubsection*{libucontext}
Libucontext is an open-sourced library that provides the ucontext.h C API. Most
notably for the project of this thesis, it can deploy on bare metal RISC-V
32-bit with newlib\footnote{Newlib is a C library intended for use on embedded
systems.}. Building the library from scratch led to some issues on my end, and
as such, the necessary files were copied and linked together with my
implementation upon building. With this, I am able to use getcontext,
makecontext and setcontext, which allows me to do the necessary context
switching described within Section~\ref{sec:Design}.


\subsection{Creating a linker script}
The linker script is used to inform the linker which parts of the file to
include in the final output file, as well as where each section is stored in
memory. As I am working on an embedded system, I have to deviate from the
default and create my own linker script. Clang uses the LLVM lld linker, which
is compatible with general linker scripts implementations of the GNU ld
linker.\cite{llvm-org-linker}

\begin{lstlisting}[caption=Memory area defined in linker script]
OUTPUT_ARCH('riscv')
ENTRY(_start)

MEMORY
{
/* Fake ROM area */
rom (rxa) : ORIGIN = 0x80000000, LENGTH = 1M
ram (wxa) : ORIGIN = 0x80100000, LENGTH = 127M
}
\end{lstlisting}
First, I must specify that I want the RISC-V architecture and designate the
entry point of the program at a function named '\_start', which I will define
later. Second, I define the MEMORY area to consist of both a writable memory
region and a read-only memory region. I name these regions 'ram' and 'rom,'
respectively. With that, I move on to define the SECTIONS element of the linker
script.

\begin{lstlisting}[caption=Linker scripts SECTIONS.]
SECTIONS {
  .text :
  {
    *(.text .text.*)
  } > rom

  .rodata :
  {
    *(.rdata)
    *(.rodata .rodata.*)
  } > rom

  .data :
  {
    *(.data .data.*)
    . = ALIGN(4);
    PROVIDE( __global_pointer$ = . + GLOBAL_STACK_SIZE );
  } > ram

  .bss :
  {
    *(.bss .bss.*)
  } > ram

  /* It is standard to have
  the stack aligned to 16 bytes*/
  . =
  _end = .;

  .stack :
  {
    . = ALIGN(4);
    PROVIDE(_stack_start = .);
    PROVIDE(_stack_top = ORIGIN(ram) + LENGTH(ram));
  } > ram
}
\end{lstlisting}
The text, rodata, data and bss sections follow the same general procedure. I
match any data for the given section, and then save it either in rom or ram. By
specifying the > rom, I tell the linker to save the given section in the ROM
section and the same is true for the > ram. From the Figure~\ref{fig:mem_layout}
a memory diagram shows how the linker places elements into memory.\footnote{ROM
  stands for Read-Only Memory, and RAM stands for Random-Access Memory.
  Generally it is not necessarily needed to split the two up as done here, but
it is a good practice to separate what can change and what cannot change in
memory.}

In the data section, I also provide a global pointer, which is used to access
global variables within our later code implementation. The global pointer is
used together with an offset to save global variables. For more customisability
this value can be passed in the Makefile.

The last section is the .stack section. I align the starting of the
\_stack\_start with 4 bytes. Generally a good rule of thumb is to guarantee
alignment to the word size, which in RISC-V is 32-bit. This is the point from
where each thread stack will be allocated. Next, I specify that the
\_stack\_top will reside at the end of the RAM section such that I can allocate
a stack for each core.

\subsection{Getting into the main function}\label{sec:get_main}
In the linker script, I specified the entry point of our program as `\_start`.
Next up is implementing this entry point in assembly. In a new assembly file, I
add the following:
\begin{lstlisting}[numbers=left, escapechar=|, caption=Assembly code for getting
to main function.]
.extern main |\label{line:header_start}|
.extern secondary_main
.globl _start
.type _start,@function
#include "../include/defines.h" |\label{line:header_end}|

_start: |\label{line:_start}|
  .cfi_startproc
  .cfi_undefined ra
  .option push
  .option norelax
  la gp, __global_pointer$
  .option pop |\label{line:pop}|
  // load _stack_top into the sp register
  la sp, _stack_top | \label{line:stack_top}|
  csrr a0, mhartid
  bnez a0, 2f | \label{line:break_zero}|
  1:
    // argc, argv is 0 and jump to main
    li  a0, 0
    li  a1, 0
    jal main | \label{line:jump_main} |
  1:
    // loop
    j 1b
  2:
    la t1, STACK_SIZE | \label{line:STACK_SIZE} |
    li t0, 0
  1:
    andi sp, sp, -16 | \label{line:align_stack} |
    beq a0, t0, 1f
    sub sp, sp, t1 | \label{line:sub_stack} |
    addi t0, t0, 1
    j 1b |\label{line:j_1b}|
  1: |\label{line:jump_secondary}|
    // argc, argv is 0 and jump to main
    li  a0, 0
    li  a1, 0
    jal secondary_main
  1:
    // loop
    j 1b
    .cfi_endproc  // This point should never be reached
\end{lstlisting}
Lines \ref{line:header_start}-\ref{line:header_end} provide the setup for the
assembly file. I specify that a main and secondary\_main label will be defined
outside of the file, that \_start is a global label and that \_start is a
function type. At last, I include the defines.h file, which includes
definitions of the STACK\_SIZE.

On lines \ref{line:_start}-\ref{line:pop} I provide call frame information
(CFI) for there being no return address and that the process begins here. Then
when initializing the global pointer, I must specify options push, norelax, and
pop as described in GNU binutils documentation. \cite{GNU_BIN} After linker relaxation, this
would produce the anticipated code:
\begin{lstlisting}
auipc gp, %pcrel_hi(__global_pointer$)
addi gp, gp, %pcrel_lo(__global_pointer$)
\end{lstlisting}

On lines \ref{line:stack_top}-\ref{line:break_zero}, I load the value of
\_stack\_top, which the linker provides through the linker script defined
previously, and save it into the stack pointer (sp) register. I read the
current machine hart identifier (mhartid), which contains a unique identifier
for each core on the processor. This is what allows for differentiation between
the different cores. Line \ref{line:break_zero} moves execution to line
\ref{line:STACK_SIZE} if the machine hart identifier is not 0. As such, only
mhartid 0 will be allowed to continue execution to line \ref{line:jump_main},
where it jumps to the main function.

All other cores continue execution at line \ref{line:STACK_SIZE}, where they
load the value STACK\_SIZE\footnote{Defined either via. the Makefile or defined
in defines.h} into the temporary register t1. I also load immediately(li) the
value 1 into the temporary register r1. Line \ref{line:align_stack} aligns the
current value in the stack pointer to 16. Afterwards, I compare the value in
register a0, which holds the value of the mhartid, to the value in t0. If they
are equal, I jump to line \ref{line:jump_secondary}, which goes to the
externally defined secondary\_main function. Otherwise, I continue on line
\ref{line:sub_stack}, where I subtract the register t1 (STACK\_SIZE) from the
value stored in the sp register. I increment the value in t0 by one, and jump
back to line \ref{line:align_stack}. With this loop, I am setting up a stack of
size STACK\_SIZE for all the different cores defined, such that I get the
desired memory layout shown at the top of Figure~\ref{fig:mem_layout}.

\subsection{Initializing the thread jobs}\label{sec:init_threads}
\begin{algorithm}
  \begin{algorithmic}[1]
    \Require threads[] \# Empty thread array
    \Procedure {Initialize\_Threads}{$A$}
    \State $depth \leftarrow MostSignificantBit(NUM\_CORES)$
    \State $idx \leftarrow 0$
    \For{$i=0, i < depth$}
    \If{$i == 0$} \label{alg:early_escape}
    \State \# Top level thread
    \State $ct \leftarrow threads[idx]$
    \State mid \leftarrow Length(A)/2
    \State $thread\_create(ct, parallel\_merge)$
    \State $ct.l = 0$
    \State $ct.mid = mid$
    \State $ct.r = Length(A)$
    \State idx++
    \State \textbf{continue}
    \EndIf
    \State $k \leftarrow 2^k$
    \For{$j= 0, j < k$}
    \State $parent\_thread \leftarrow threads[(idx - 1)/2]$
    \State $ct \leftarrow thread[idx]$
    \If{$j \% 2 == 0$} \label{alg:left_right}
    \State $ct.l = parent\_thread.l$
    \State $ct.r = parent\_thread.mid$
    \Else
    \State $ct.l = parent\_thread.mid$
    \State $ct.r = parent\_thread.r$
    \EndIf
    \State $ct.mid = ct.l + (ct.r - ct.l)/2$ \label{alg:left_right_end}
    \If{$i == depth-1$} \label{alg:depth_check}
    \State \# Just regular merge sort
    \State thread\_create(ct, merge sort)
    \Else
    \State \# Merge given section
    \State thread\_create(ct, parallel\_merge)
    \EndIf
    \State idx++
    \EndFor
    \EndFor
  \EndProcedure
  \end{algorithmic}
  \caption{Initialization of the threads}\label{algo:threads}
\end{algorithm}
A pseudocode implementation for initializing the parallel merge sort algorithm
is provided in Algorithm~\ref{algo:threads}. It should be noted that it is
assumed NUM\_CORES is a multiple of 2. Consequently, I can calculate the depth
desired for our merge tree by employing the number of cores. This may be
achieved by taking $\log_2(NUM\_CORES)$ or, equivalently, by examining the most
significant bit of NUM\_CORES since it is a multiple of 2. At
line~\ref{alg:early_escape}, an early escape mechanism is implemented which
takes care of the single thread that will have no parent thread. If I am not
positioned at the very top level of the merge tree, I assign the variable 'k'
as the number of threads for the given level and partition the list to create
threads targeting a specific sublist. Lines \ref{alg:left_right} -
\ref{alg:left_right_end} address the task of determining which subsection of the
list each thread is responsible for. Finally, line~\ref{alg:depth_check} ensures
that I do not generate more active threads than what the available number of
cores can effectively handle. Upon completion of the algorithm, a threads array
will be produced containing $2 \cdot NUM\_CORES - 1$ different threads, each
assigned with a specific subsection of the list to either merge or perform
merge sort on.

\subsection{Reducing synchronization}
\begin{figure}
  \begin{center}
    \includegraphics[width=0.85\textwidth]{figures/mergesort.png}
  \end{center}
  THREADS := [(A, 0, 8, MERGE), (A, 0, 4, MERGE), (A, 4, 8, MERGE), (A, 0, 2,
  MERGESORT), (A, 2, 4, MERGESORT), (A, 4, 6, MERGESORT), (A, 6, 8, MERGESORT)]
  \caption{Example of partitioning a random list using
  Algorithm~\ref{algo:threads}}\label{fig:mergesort}
\end{figure}

The idea behind initializing the threads array, as described in
Section~\ref{sec:init_threads}, is that I can reduce the need for
synchronization between threads once the partitioning of the list is done. An
example can be seen in Figure~\ref{fig:mergesort} with the corresponding
finished THREADS array. This example would be on a system with 4 cores. As seen
in Section~\ref{sec:get_main}, each core has a unique \textbf{id}, mhartid, from 0 to 3.
With this, each core can index into the list with the number of jobs it has
already completed and the length of the array list such that:
\begin{align}
  index &= LENGTH(A) - \text{NUM\_COMPLETED\_JOBS}\cdot id
\end{align}
With this, each thread is capable of retrieving a thread job without having to
consider the state of any other thread, as the job is preassigned during the
initialization. Each merge must still wait on the child thread (the direction of
the arrow in Figure~\ref{fig:mergesort}) to finish before starting a merge.
However, as there is no queue, there is no need for synchronization beyond that
between the threads.


