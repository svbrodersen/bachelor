\section{Implementation}
The implementations created as part of this bachelor thesis aimed
to make use of the LLVM compiler infrastructure.
LLVM is a collection of modular and reusable compiler and tool chain technologies,
most notably for this project is the clang compiler. Furthermore, QEMU will be used extensively while testing the
implementations.

\subsection{Dependencies}
\subsubsection{QEMU}
\begin{lstlisting}[caption=Installing QEMU, float=*, label=lst:qemu_install]
git clone git clone https://github.com/qemu/qemu  # Clone the qemu repo
./configure --target-list=riscv32-softmmu  # Configure the 32-bit RISC-V target
make -j $(nproc)  # build the project with all num cores jobs
sudo make install
\end{lstlisting}
Following the instructions by RISC-V's getting started guide
we can build the QEMU RISC-V system emulators by running the code
provided in Listing~\ref{lst:qemu_install}\cite{RISC-V_GS}.
\subsubsection{Installing LLVM compiler infrastructure}
\begin{lstlisting}[caption=Installing LLVM compiler infastructure with RISC-V
32-bit as native target., float=*, label=lst:llvm_install]
# Dependencies
sudo apt-get -y install \
  binutils build-essential libtool texinfo \
  gzip zip unzip patchutils curl git \
  make cmake ninja-build automake bison flex gperf \
  grep sed gawk python bc \
  zlib1g-dev libexpat1-dev libmpc-dev \
  libglib2.0-dev libfdt-dev libpixman-1-dev

# Installing the RISC-V-gnu-toolchain with llvm support
git clone https://github.com/riscv-collab/riscv-gnu-toolchain  # clone
riscv-gnu-toolchain
cd riscv-gnu-toolchain  # change directory
./configure --prefix=/opt/riscv --with-arch=rv32gc -disable-linux --enable-llvm
# prefix is install path used by llvm
sudo make -j$(nproc)
cd ..
popd
\end{lstlisting}
When it comes to clang there are two methods of installing, that are relevant to
this project. If running on a Debian based system, then you can simply install
llvm-tools package. The issue with this approach is that the general build is
for use with the current system installation is on, which unless you are running
a RISC-V computer architecture natively will lead to issues when trying to cross
compile if the given targets use any of the standard libraries, such as
freeRTOS. A fix to this issue is to explicitly tell clang to make use of the
RISC-V gnu tool chain on every compilation.

The second approach is to build LLVM with the RISC-V 32-bit target as the native
target. This approach is documented in Listing~\ref{lst:llvm_install}. After
installation it is important to add both clang build and RISC-V gnu tool chain to
PATH. However, adding the following flags to compilation
should lead to the same results, although the second approach is used throughout
this project.
\begin{itemize}
  \item --sys-root={Path to RISC-V install}/riscv64-unknown-elf
  \item --taget=riscv32
  \item --gcc-toolchain={Path to RISC-V install}
\end{itemize}


\subsection{Creating a linker script}
The linker script is used to tell the linker which parts of the file to include
in the final output file, as well as where each section is stored in memory. As
we are working on an embedded system, we have to stray from the default and
create our own linker script. The clang uses the LLVM lld linker, which is compatible
with the general linker scripts implementations of the GNU ld linker \cite{llvm-org-linker}.
Thus, we can make use of the GNU ld manual for modifying the linker script in freeRTOS for our bare metal application instead of writing the entire thing from scratch \cite{GNU-linker}.

\begin{lstlisting}
OUTPUT_ARCH('riscv')
ENTRY(_start)

MEMORY
{
/* Fake ROM area */
rom (rxa) : ORIGIN = 0x80000000, LENGTH = 1M
ram (wxa) : ORIGIN = 0x80100000, LENGTH = 127M
}
\end{lstlisting}
First, we must specify that we want the RISC-V architecture and designate the entry point
of the program at a function named '\_start,' which we will define later.
Second, we define the MEMORY area to consist of both a writable memory region and a read-only
memory region. We name these regions 'ram' and 'rom,' respectively. With that we move on to define the SECTIONS element of the linker script.

\begin{lstlisting}
SECTIONS
{
.text : ALIGN(CONSTANT(MAXPAGESIZE))
{
  *(.text .text.*)
}

.rodata : ALIGN(CONSTANT(MAXPAGESIZE))
{
  *(.rdata)
  *(.rodata .rodata.*)
}

.data : ALIGN(CONSTANT(MAXPAGESIZE))
{
  *(.data .data.*)
  /*RISCV convention to have __global_pointer
  aligned to 8 bytes*/
  . = ALIGN(8);
  PROVIDE( __global_pointer$ = . + 0x800 );
}

.bss : ALIGN(CONSTANT(MAXPAGESIZE))
{
  *(.bss .bss.*)
}

/* It is standard to have
the stack aligned to 16 bytes*/
. = ALIGN(16);
_end = .;

.stack : ALIGN(CONSTANT(MAXPAGESIZE))
{
  _stack_top = ORIGIN(ram) + LENGTH(ram);
}
}
\end{lstlisting}
The text, rodata, data and bss sections follows the same general procedure. We align the
section to the maximum size of a page, and the match all the data which we care about for
the given sections. I have opted to disregard specifying where the linker has to save
all the data, and instead opted to let the linker itself find a suited location looking at
the attributes we gave to memory previously.
In the data section, we also provide a global pointer, which is used to access global variables
within our later code implementation. \footnote{In general the global pointer is used together
with an offset in much the same manner as a stack pointer and offset.}

Then I align the end with 16 bytes as is the custom. Reason this is moved outside the stack
is that I use the \_end variable later, and as such it should be aligned as well.
Then within the stack section we define the \_stack\_top as being the end of the random access
memory section (ram), as the stack grows downwards.

\subsection{Getting into the main function}
In the linker script we specified the entry point of our program as \_start. Next up
is implementing said entry point in assembly. Within a new assembly file we add the following.
\begin{lstlisting}
.extern main
.section .init
.globl _start
.type _start,@function

_start:
  .cfi_startproc
  .cfi_undefined ra
  .option push
  .option norelax
  la gp, __global_pointer$
  .option pop
  // load _stack_top into the sp register
  la sp, _stack_top
  add s0, sp, zero

  // argc, argv, envp is 0 and jump to main
  li  a0, 0
  li  a1, 0
  li  a2, 0
  jal main
  .cfi_endproc // We end the proccess
\end{lstlisting}
First we specify that externally there will be implemented a main entry point. Next we tell the
linker to save the following code in the .init section and initialize a global label \_start,
and note that it is a function.

Next the \_start is defined, and define .cfi\_startproc such that we have an entry in the
.eh\_frame. Next we define the return address register(ra) as being undefined, as we are in
the start of the entire program. Since the linker usually relaxes addressing sequences to
shorter GP-relative sequences when possible, the initial load of GP must not be relaxed
\cite{GNU_BIN}. However, we do not need the same for loading the \_stack\_top into the sp
register, and then also save it in the s0 register, which stores the frame pointer.
Then the last step is loading 0 into argc, argv and envp and the jump to the externally
defined main function.

\subsection{Printing with UART}
\begin{lstlisting}[float=*, label=lst:putchar, caption=Implementation of putchar of stdarg lib]
#include <stdint.h>
#define UART_ADDR 0x10000000
#define LCR 0x03          // Line control register
#define LSR 0x05          // Line status register
#define FCR 0x02          // FIFO control reigster
#define RBR 0x00          // Receiver buffer register
#define IER 0x01          // Interrupt enable register
#define LSR_THRE 0b110000 //
void uart_init(void) {
  volatile uint8_t *ptr = (uint8_t *)UART_ADDR;

  // Set word length to 8 (LCR[1:0])
  *(ptr + LCR) = 0b11;

  // Enable FIFO (FCR[0])
  *(ptr + FCR) = 0b1;

  // Enable receiver buffer interrupts (IER[0])
  *(ptr + IER) = 0b1;
}

static void uart_put(uint8_t c) { *(uint8_t *)(UART_ADDR + RBR) = c; }
static uint8_t uart_get(uintptr_t addr) { return *(uint8_t *)(addr); }

void putchar(unsigned char c) {
  volatile uintptr_t ptr = (uintptr_t)UART_ADDR;
  // make sure there is nothing else in FIFO
  while ((uart_get(ptr + LSR) & LSR_THRE) == 0) {
    // do nothing
  }
  // add the char to reciver buffer register
  uart_put(c);
}
\end{lstlisting}
As we are working with bare metal, the standard printf function will not work. However, QEMU
allows the use of universal asynchronous receiver / transmitter (UART) protocol, which allows
us to implement a printf function, that writes to the terminal without the need for the QEMU
GUI interface. Furthermore, there exists open source bare-metal versions of the printf function,
which makes the effort of printing a lot easier. One such example is Georges Menie's stdarg,
which depends on a single function called putchar, which has to take a character and place it
somewhere. One such implementation can be seen in Listing~\ref{lst:putchar}. Again this code is
a modified version of what is seen in freeRTOS. From the previous dtc file created
in Section~\ref{sec:sys-info} there is also information regarding the UART configuration.
\begin{lstlisting}
serial@10000000 {
  interrupts = <0x0a>;
  interrupt-parent = <0x03>;
  clock-frequency = "\08@";
  reg = <0x00 0x10000000 0x00 0x100>;
  compatible = "ns16550a";
};
\end{lstlisting}
First it states the address of the uart is, 0x10000000, and it states that it is ns16550a compatible \cite{uart}. In Listing~\ref{lst:putchar} you can see the needed implementations for the
uart to work, which is a slightly modified version of what is seen in freeRTOS. First,
we define the register values as stated in the uart documentation, and initialize the uart.
Next, we define af uart\_put and uart\_get function to read from and write to addresses. With this we are able to define the putchar function. It works by busy waiting until the FIFO queue is
empty, and then gives the character to the reciver buffer register.

\subsection{libucontext}
