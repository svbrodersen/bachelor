\section{Conclusion}
\subsection{Summary of results}
This thesis discusses the viability of implementing custom-made solutions for
computational storage devices. In the first part of the thesis, the problem is
presented along with the background. Next, the approach of implementing a
bare-metal merge sort application on the QEMU virtual machine is proposed. A
design for a merge sort algorithm that does not make use of any queue is
suggested to reduce the need for synchronization. With this, a method of memory
allocation is discussed and the design is carried out making use of context
switching. The implementation of the merge sort algorithm is tested with varying
list sizes as well as a varying number of cores utilized. Finally, possible
improvements are presented in future work and left as open research questions.

\subsection{Takeaways}
In Section~\ref{sec:problem} two questions were proposed and aimed to be
answered throughout this thesis. As such, these will build the foundation of the
following conclusion.

\subsubsection*{What computation should be handled by a storage device?}
Within the context of an academic thesis, it is demonstrated that intricate
computations, such as sorting, are indeed feasible to implement on a bare-metal
storage device. However, this endeavor does not come without its share of
complications. These include creating a custom linker script and devising custom
assembly scripts, in addition to other obstacles left unmentioned within the
scope of this thesis. These challenges may deter the average consumer;
nonetheless, when viewed from the perspective of large-scale data centers, the
potential for computational storage remains bright. As the CPU increasingly
becomes a bottleneck in data transfer between a storage device and its host,
computational storage has the potential to go beyond just sorting capabilities.
Consequently, this thesis proposes that it is not the complexity of the
computation at hand but rather the magnitude of demand from large data centers
that will ultimately determine the success of the computational storage device.


\subsubsection*{Is it feasible to implement such a computation on a bare-metal
RISC-V processor?}
The bare-metal approach presents numerous challenges that are not encountered
when utilizing an underlying operating system. Despite these limitations, with
perseverance, it is possible to overcome them and optimize a system specifically
tailored to meet certain demands. In the context of large data centers,
developing a custom computational storage device is both viable and could be
advantageous. Nevertheless, it remains uncertain whether the performance
enhancements justify the investment in implementing such systems.


