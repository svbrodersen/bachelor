\section{Choosing Stack Sizes}\label{ap:stack}
When choosing the stack size one can make a calculate guess on the size needed.
Firstly we split it into the three different stack types; Thread, Core and
Global.

\subsection*{Core stack}
The Core stack is the simplest. Generally, the entire computation done on a
given core will always be the same, no matter the size of the list. This is due
to how the context switching works. Whenever we switch context, we are using a
different stack, and since we always do a context switch before copying data for
a merge operation, then the stack size needed will generally be the same. From
testing I found that a STACK\_SIZE of 2048 bytes was enough.

\subsection*{Thread stack}
Each thread job will execute at least one merge operation where the merge
operation has to copy the entire section it currently is working on. Thus, in
the worst case a thread will have to copy the entire list when doing a merge
operation. Following this rule of thumb, we have:
\begin{align}
  \text{THREAD\_STACK\_SIZE} &= \text{LIST LENGTH} * 4 + C
\end{align}
Where 4 denotes the size of an integer in bytes, and C is some constant extra
size needed for all the default variables. Throughout my testing, I found that
having a C value of 1024 generally was enough for the THREAD\_STACK\_SIZE.

\subsection*{Global Stack}
The global stack stores both the entire list and the thread jobs. Within GDB I
printed the size of the thread\_t struct, which told me its size was 848 bytes.
Thus the Global Stack size generally can be set to:
\begin{align}
  GLOBAL\_STACK\_SIZE &= \text{LIST LENGTH} * 4  \\
                      &+ (2 * \text{NUM\_CORES} - 1) * 848  + C
\end{align}
Where 4 denotes the size of an integer in bytes, 848 is the size of the struct
in bytes, $(2 * \text{NUM\_CORES} - 1)$ is the number of thread jobs and C is
some constant needed for the different variables. Again a value of 1024 for C
was generally enough.

