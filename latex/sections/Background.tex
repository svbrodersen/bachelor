\ssection{Background}
\subsection{Accelerator-based Computer Architecture}
The concept of offloading is nothing new. In a team where each member
specializes, it seems logical when discussing our day-to-day work environment.
Effective communication between entities while focusing on what we do best seems
logical when speaking about work. However, when it comes to computer
architecture, we heavily rely on the Central Processing Unit (CPU). The goal of
an accelerator would be to offload the CPU, by providing a separate
substructure designed with different objectives than the CPU itself.
With this design, the substructure can be optimized for its
specific tasks, often leading to both performance increases and less energy
consumption\cite{AA}. Prominent examples of accelerators include the Graphics
Processing Unit (GPU), which is a major component in most computers today. The
goal of this thesis is to examine the feasability of following a similair design
pattern in creating computational storage devices.

\subsection{RISC-V}
Reduced Intruction Set Computer(RISC), more specifically the fifth version
(RISC-V). Is an Instruction Set Architecture(ISA), that aims to make the process
of making custom processors targeting a variety of end applications more
feasible. Previous ISAs have often been created by private companies, which
leads to patents and a need for a license to develop a specialised processor.
These licenses could often take months to negotiate without mentioning the large
sum of money involved. It is assumed that creating a free and open sourced ISA
could reduce the barrier of entry and greatly increase innovation along with
afforability\cite{ISAfree}.

RISC-V aims to provide a small core of instructions which compilers, assemblers,
linkers and operating systems can generally rely on, while still being
extendable for more specialised accelerators. In RISC-V there  are two primary
base integer variants, RV32I and RV64I, whcih provide the 32-bit and 64-bit
user-level address spaces respectively. However, RISC-V is already in the works
with a RV128I variant which would provide the foundation needed for a 128-bit
user address space in the future. In gereral, RISC-V provides standard and
non-standard extensions, where standard extensions should not conflict with
other standard extensions, and the non-standard extensions are more highly
specialised.


\subsection{Toolchain}
\begin{lstlisting}[caption=Installing QEMU, float=*, label=lst:qemu_install]
git clone git clone https://github.com/qemu/qemu  # Clone the qemu repo
./configure --target-list=riscv32-softmmu  # Configure the 32-bit RISC-V target
make -j $(nproc)  # build the project with all num cores jobs
sudo make install
\end{lstlisting}
\subsubsection{QEMU}
QEMU is a system emulator, which has the capabilities of emulating both a 32-bit
and 64-bit RISC-V processor \cite{QEMU}. With QEMU I am able to create code intented
for a processor running the RISC-V instruction set even if my development
environment is running a different ISA. For the puposes of this thesis it is the
RISC-V 32-bit version of the qemu virtual machine that will be used.
Following the instructions by RISC-V's getting started guide
we can build the QEMU RISC-V system emulators by running the code
provided in Listing~\ref{lst:qemu_install}\cite{RISC-V_GS}.

\subsubsection{RISC-V-GNU-Toolchain}

