\section{Background}
\subsection{Accelerator-based Computer Architecture}\label{sec:ACA}
The notion of offloading has long been established in specialized teams, where
each member focuses on their area of expertise. This concept seems inherently
logical when discussing day-to-day work environments. Effective communication
between entities, with an emphasis on performing tasks best suited to our
skills, appears to be the foundation of efficient collaboration. Contrary,
computer architecture relies heavily on the CPU for executing various
operations. An accelerator serves as a separate substructure designed with
distinct objectives compared to the CPU itself. By offloading the CPU,
accelerators can optimize performance and reduce energy consumption\cite{AA}. A
prime example of an accelerator is the Graphics Processing Unit (GPU), a crucial
component in contemporary computers. This thesis aims to explore the feasibility
of adopting a similar design approach for creating computational storage
devices.



\subsection{RISC-V}
Reduced Instruction Set Computing (RISC), particularly its fifth iteration,
RISC-V, represents an Instruction Set Architecture (ISA) designed to simplify
the development of custom processors for various applications. Unlike
proprietary ISAs created by private companies, RISC-V offers a free and
open-source solution that minimizes intellectual property concerns and reduces
entry barriers, promoting innovation and affordability in processor
development\cite{ISAfree}.

RISC-V aims to provide a small core of instructions which compilers, assemblers,
linkers, and operating systems can generally rely on, while still being
extendable for more specialized accelerators. In RISC-V there are two primary
base integer variants, RV32I and RV64I, which provide the 32-bit and 64-bit
user-level address spaces respectively. However, RISC-V is already in the works
with a RV128I variant which would provide the foundation needed for a 128-bit
user address space in the future. In general, RISC-V provides standard and
non-standard extensions, where standard extensions should not conflict with
other standard extensions, and the non-standard extensions are highly
specialized.

With the rise of ARM\footnote{short for Advanced RISC Machine. ARM is another
RISC based ISA. Difference is it is proprietary.} based machines with comparable
and in some cases better performance than that of a Complex Instruction Set
Computing(CISC) alternative\cite{Power_Struggle}. RISC-V aims to provide the
same benefits in an open sourced environment. With this RISC-V, more
specifically the 32-bit version, was chosen as the ISA for development in this
thesis.

\subsection{Computational Storage}
Computational storage can be seen as a subsection of Accelerator-based Computer
architecture. Firstly, it aims to offload the host processor as described in
Section~\ref{sec:ACA} by providing a secondary processors optimized for specific
computational tasks. Secondly, it aims to reduce data movement between the
storage device and the host processor. This would allow the read and writes to
be distributed among multiple RAM sections rather than a single processor. This
could be an integral part of the issues presented in Section~\ref{sec:context},
as a computational storage device would be scalable with the ever-growing need
for larger volumes of data processing.


\subsection{Toolchain}
\subsubsection*{QEMU}
QEMU short for quick emulator system emulator, which has the capabilities of
emulating both a 32-bit and 64-bit RISC-V processor \cite{QEMU}. With QEMU I am
able to create code intended for a processor running the RISC-V instruction set
even if my development environment is running a different ISA. For the purposes
of this thesis it is the RISC-V 32-bit version of the QEMU virtual machine that
will be used. The RISC-V 32-bit virtual machine is compatible with the standard
extensions, and throughout this thesis the following standard extensions will be
used. The Control and Status Register (Zicsr), Atomic Instructions(a), Integer
Multiplication and Division (m) and Compressed Instructions(c).

\subsubsection*{LLVM and RISC-V GNU Toolchain}
The LLVM project is a collection of reusable compiler and toolchain
technologies. Most notably for the context of this thesis clang. Clang is a gcc
compatible frontend compiler, which aims to provide fast compile times and low
memory use. In tandem with the LLVM compiler back end, clang provides a
library-based architecture such that the compiler can work together with other
tools. This allows for the use of more sophisticated development environments
such as an Language Server Protocol(LSP). Generally clang also provides more
sophisticated error reports making the overall debugging easier.

At the time of writing, LLVMs LLDB debugger's connection to the RISC-V QEMU
machine was unable to connect probably. Consequently, the GNU gdb debugger for
the RISC-V target was compiled for use in the implementation section. For less
headache with changing platforms, the same was done for the LLVM clang compiler.
Additionally, the RISC-V GNU toolchain supplies essential standard library
header files, enabling basic algorithm implementations, such as
memcpy\footnote{Used by default when copying one memory buffer to another.}, for the
compiler.

