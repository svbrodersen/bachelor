\section{Evaluation}

\subsection{Testing}
Testing has been accomplished by providing a random\_numbers.py file. With 3
seperate integer parameters it creates a random list using pythons standard
random library. The inputs include a lower and upper bound for the list to
generate, together with the length of the list. Once the list is create, running
make will create a .elf file, which contains the parallel mergesort algorithm
with the unsorted list hardcoded within. Once the .elf file is loaded on a
RISC-V processor, it will immediately begin sorting the hardcoded list. The
implementation also needs the value NUM\_CORES defined within the Makefile,
where it both defines a constant NUM\_CORES for the .elf file to use, and the
same value is used for running the QEMU virtual machine.

This gives the following work flow for creating and running a test:
\begin{itemize}
  \item Run random\_numbers.py to generate alist.c with an unsorted list.
  \item Change the NUM\_CORES variable in the Makefile to the desired number of
    cores.
  \item run "make test" to generate the .elf file and host qemu.
  \item run "python3 validate.py", which takes the output of the non-sorted
    array, sorts it via. pythons inbuilt sort function, and compares it to the
    output generate by running the .elf file
\end{itemize}

Within the directory there exists a .gdbinit file, which takes care of
connecting to the qemu session. It also provides a break point in the main
function.

\subsection{Validation}
\begin{table}
  \caption{Table of tests run}\label{tab:tests}
  \begin{center}
    \begin{tabular}[c]{l|l|l|l|l|l|l}
      & \multicolumn{6}{c}{NUM\_CORES}\\
      \cline{2-7}
       Range & 1 & 2 & 3 & 4 & 8 & 16 \\
      \hline
      -100:0:100 & false & true & false & true & true & true \\
      \hline
      0:100:100 & false & true & false & true & true & true \\
      \hline
      -50:50:100 & false & true & false & true & true & true
    \end{tabular}
  \end{center}
\end{table}

For demonstration the elf file first prints out the array it was given to
stdout, and then again prints it once the array, once it has been sorted in
place.


\subsection{Future work}
NUM CORES HAS TO BE A POWER OF TWO


