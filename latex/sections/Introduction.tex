\ssection{Introduction}
\subsection{Context}\label{sec:context}
Data centers are becoming an ever integral part of the IT world. Whether it is
Google's cloud platform, Microsoft Azure or Amazons web services, news about a
new data center seem like daily occurences. With such scale there is an ever
growing need for custom solutions and cutting edge technologies to both reduce
power consumption and improve overall performance.

Historically, solid state drives(SSDs) were a drop in replacement for the
magnetic disks of the past. They would implement a similair interface allowing
for seamless integration. But the use of SSDs came with multiple improvements
over the magnetic disks of the past, which were hindred by said interface. As
such there was a rapid movement towards Open-channel SSDs, that do not have a
firmware Flash Translation Layer, and instead leaves the management of the
physical SSD to the computer's operating system. This solves the issue mentioned
previously, but instead introduces further data transfering between the CPU and
the SSD. However in recent years, the discrepancy between a storage devices
READ and WRITE operations and a CPUs ability to perform READ and WRITE memory
operations has been ever increasing. If this trend keeps up, the CPU will soon
be a bottleneck for performance.

A solution to the problem would be to offload the CPU, and provide computation
at the SSD. Such a solution has been described as a computational storage
device(CSD). Such a solution would provide implementations of the most often use
data manipulations. Whether that would be indexing into an SSD or more complex
manipulations such as sorting. Within this thesis the issue of implementing a
high performance sorting running on a stand-alone bare metal processor has been
investigated.


\subsection{Problem}



\subsection{Approach}

\subsection{Contribution}

\subsection{Related work}


