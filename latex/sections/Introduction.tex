\ssection{Introduction}
\subsection{Context}\label{sec:context}
Data centers are becoming an ever integral part of the IT world. Whether it is
Google's cloud platform, Microsoft Azure or Amazons web services, news about a
new data center seem like daily occurences. With such scale there is an ever
growing need for custom solutions and cutting edge technologies to both reduce
power consumption and improve overall performance.

Historically, solid state drives(SSDs) were a drop in replacement for the
magnetic disks of the past. They would implement a similair interface allowing
for seamless integration. But the use of SSDs came with multiple improvements
over the magnetic disks of the past, which were hindred by said interface. As
such there was a rapid movement towards Open-channel SSDs, that do not have a
firmware Flash Translation Layer, and instead leaves the management of the
physical SSD to the computer's operating system. This solves the issue mentioned
previously, but instead introduces further data transfering between the CPU and
the SSD. However in recent years, the discrepancy between a storage devices
READ and WRITE operations and a CPUs ability to perform READ and WRITE memory
operations has been ever increasing. If this trend keeps up, the CPU will soon
be a bottleneck for performance.

A solution to the problem would be to offload the CPU, and provide computation
at the SSD. Such a solution has been described as a computational storage
device(CSD). Such a solution would provide implementations of the most often use
data manipulations. Whether that would be indexing into an SSD or more complex
manipulations such as sorting. Within this thesis the issue of implementing a
high performance sorting running on a stand-alone bare metal processor has been
investigated.


\subsection{Problem}
For computational storage to be a viable solution to the ever growing need of
massive amount of data computations, it first must be investigated whether
implemneting a processor deisgned for such a goal is feasible. As such we are
left with the following open questions. (1) What computation should be handled
by a storage device? (2) Is it feasible to implement such a computation on a
bare metal processor?

\subsubsection*{What computation should be handled by a storage device?}
Although there are multiple cases of large data transfers between a CPU and an
SSD, one of the more prominent is that of sorting a given array. Sorting plays
an integral part in multiple programming scenarios. From being an integral part
of much searching algorithm to use in data science fast sorting is a necessicty
for fast performance. With a running time of O(n log n), merge sort was the
algorithm chosen for further investigation. Not only that, but parallel versions
of the merge sort algorithms should be possible in theory.

\subsubsection*{Is it feasible to implement such a computation on a bare metal
RISC-V processor?}
As the main goal is to offload the main CPU, we must investigate, whether it is
at all possible to create a high performance sorting algorithm without the need
of an underlying operating system.

\subsection{Approach}
For this thesis an experimental approach was taken. First, a feasible design
developed for implementing on a bare metal processor is inroduced. Secondly, an
implementation of said designed is presented. Third, the viability and validity
of the implementation is evaluated. Atlast, shortcomings and proposed
further research is presented. These implementations will be carried out on a
QEMU virtual machine where the code is loaded via the general loader.

\subsection{Contribution}


\subsection{Related work}


