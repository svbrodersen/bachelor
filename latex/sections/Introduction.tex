\ssection{Introduction}
\subsection{Context}\label{sec:context}
Data centres are becoming increasingly essential in the IT sector. Whether it is
Google's cloud platform, Microsoft Azure, or Amazon's web services, news about
new data centres seems like a daily occurrence. With such scale comes an
ever-growing need for custom solutions and cutting-edge technologies to both
reduce power consumption and improve overall performance.

Historically, solid-state drives (SSDs) were a drop-in replacement for the
magnetic disks of the past. They would implement a similar interface, allowing
for seamless integration. But the use of SSDs came with multiple improvements
over the magnetic disks of the past, which were hindered by said interface. As
such, there was a rapid movement towards Open-channel SSDs that do not have a
firmware Flash Translation Layer and instead leave the management of the
physical SSD to the computer's operating system. This solves the issue mentioned
previously but introduces further data transferring between the CPU and the SSD.
However, in recent years, the discrepancy between a storage device's READ and
WRITE operations and a CPU's ability to perform READ and WRITE memory operations
has been ever increasing. If this trend continues, the CPU will soon become a
bottleneck for performance in the data centers.

A solution to the problem would be to offload the CPU and provide computation at
the SSD level. Such a solution has been described as a computational storage
device (CSD). This would involve implementing the most commonly used data
manipulations, such as indexing into an SSD or more complex manipulations like
sorting. Within this thesis, the issue of implementing a high-performance
sorting algorithm running on a stand-alone bare metal processor has been
investigated.


\subsection{Problem}
For computational storage to be a viable solution for meeting the ever-growing
demand for massive data computations, it is essential to investigate whether
implementing a processor designed for such a purpose is feasible. Consequently,
several open questions remain unanswered. (1) What type of computation should be
performed by a storage device? (2) Is it possible to implement such computation
on a bare-metal processor?

\subsubsection*{What computation should be handled by a storage device?}
Although there are multiple cases of large data transfers between a CPU and an
SSD, one of the more prominent is that of sorting a given array. Sorting plays
an integral part in multiple programming scenarios. From being an integral part
of many searching algorithms to its use in data science, fast sorting is a
necessity for fast performance. With a running time of O(n log n), merge sort
was the algorithm chosen for further investigation. Not only that, but parallel
versions of the merge sort algorithm should be possible on bare-metal.

\subsubsection*{Is it feasible to implement such a computation on a bare metal
RISC-V processor?}
As the main goal is to offload the primary CPU, we must investigate whether it
is at all possible to create a high-performance sorting algorithm without the
need of an underlying operating system.

\subsection{Approach}
For this thesis, an experimental approach was taken. First, a feasible design
developed for implementing on a bare metal processor is introduced. Secondly, an
implementation of said designed is presented. Third, the viability and validity
of the implementation is evaluated. Lastly, shortcomings and proposed further
research are presented. These implementations will be carried out on a QEMU
virtual machine where the code is loaded via a general loader.

\subsection{Contribution}


\subsection{Related work}


